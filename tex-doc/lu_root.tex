%% Root
\documentclass[letter,12pt]{article}
\usepackage{fullpage}  %To make the art22icle go to the full page
\usepackage[letterpaper, hmargin=0.8in,top=0.62in,bottom=0.65in]{geometry} %To set the margins. hmargin is for the left and right margin and vmargin is for the top and bottom margin
\usepackage{setspace}  %To use double spacing
\usepackage{graphicx}  %To include graphics
\usepackage{subfig}    %To place figures side-by-side
\usepackage{multirow}  %To include multirow tables
\usepackage{booktabs}  %To make professional looking tables
\usepackage{siunitx}   %To align table columns by a decimal point
\usepackage{rotating}  %To make sideways tables
\usepackage{float}     %To float figures and tables
\usepackage{amsmath}
\usepackage[small,compact]{titlesec} %To reduce the size and space between sections
\usepackage{verbatim} %To allow for verbatim text and multiline comments
\usepackage{footmisc}  %This package allows us to label footnotes.
\usepackage{caption}  %For a top and bottom caption or multiple captions
\usepackage[toc,page]{appendix} %To offset appendices
\usepackage{subfiles} %For including subfiles
\usepackage{xspace} %For adding space after defining variables
\usepackage{multirow} %For multiple rows in a table
\usepackage{changepage} %To change the margins of the page on the fly,
\usepackage{tabularx}
\usepackage{amssymb} % For checkmarks 
\usepackage[round]{natbib}
\usepackage[colorlinks = true,
            linkcolor = blue,
            urlcolor  = blue,
            citecolor = blue,
            anchorcolor = blue]{hyperref} % from http://tex.stackexchange.com/a/61016/16412
%%bold caption labels for figures and table
\captionsetup[table]{labelfont=bf}
\captionsetup[figure]{labelfont=bf}
\usepackage{amsmath} %Helps with typesetting math
\DeclareMathOperator*{\argmin}{\arg\!\min}

\newenvironment{nospace}%
{\noindent\ignorespaces}%
{\par\noindent%
\ignorespacesafterend}

%%https://kawahara.ca/latex-how-to-programmatically-change-the-path-of-your-figures/
%% Figure path 
\def \FigPath {../output-plots}
%% Table path 
\def \TexTablePath {../output-tex}


\begin{document}
%\renewcommand{\harvardurl}{URL: \url}	%Necessary to use the harvard package
\title{Highly Disaggregated Land Unavailability}

%The thanks puts the school and email in a footnote
\author{Chandler Lutz \\ University of North Carolina, Charlotte \\ \\ Ben Sand \\ York University \\ \\ \\ Land Unavailability Data: \\ \url{https://github.com/ChandlerLutz/LandUnavailabilityData} \\ \\
} \date{\today}

\maketitle
\thispagestyle{empty}

\begin{abstract}
Standard empirical proxies contradict the canonical prediction that supply constraints amplify house price cycles. We resolve this puzzle by constructing the Land Unavailability--Machine Learning (LU-ML) Indices using high-resolution satellite imagery. Unlike existing measures, our indices capture the nonlinear and heterogeneous impacts of physical geography. We document two main facts: (1) physical constraints are a key determinant of cross-sectional price dynamics, driven largely by the intensive margin (steep slopes); and (2) looser supply constraints significantly mitigate the price effects of demand growth, overturning the ``null result'' in recent literature. 
\end{abstract}

%Double space the document
\doublespace
%\onehalfspacing


\emph{JEL Classification: R30, R31, R20};

\emph{Keywords: Land Unavailability, Supply Constraints, House Prices, Entrepreneurship}

%Set the above display skip to zero so there is no extra space above an equation -- only for double space
\setlength\abovedisplayskip{2pt}

%Set the below display skip to zero so there is not extra space below equations -- only for double space
\setlength\belowdisplayskip{2pt}

\clearpage
\setcounter{page}{1}

\input{lu_main.tex}

\clearpage
%Insert the bibliography
\begin{singlespace}
%\setlength{\bibsep}{0pt plus 0.5ex}
\bibliographystyle{abbrvnat}
\bibliography{00-lu,00-lu_data,00-lu_news_articles}
\end{singlespace}

%Add the appendix
\clearpage
\begin{appendix}
  
%Add the figures
\clearpage
%% figures

\thispagestyle{empty}

\begin{figure}[h!]
\centering
\caption{\textbf{The Natural Amenity Premium in US Housing Markets, 1976--2024}}
\label{fig:gmaps_amenities_hp_regs}

\centerline{\includegraphics[scale=1]{\FigPath /gmaps_amenity_hp_regs}}
\caption*{\footnotesize{\textit{Notes:} Regressions of the year-over-year (YoY) log change in CBSA house prices on Google Maps Natural Amenity Demand across housing market cycles and for the full sample by house price dataset. The CBSA-level Google Maps Natural Amenity Demand Index is the sum of the log Google Maps Amenity Demand Indices for ``Hiking,'' ``Natural Amenity Recreation,'' ``Water Amenities,'' and ``Viewpoints.'' See \citet{Lutz2025amenities} for further computational details. We regress the log annual change in house prices for CBSA $i$ on the interaction Google Amenity Demand Index with housing market cycle $c$ using: $\Delta \ln(P_{i,t}) = \alpha_t + \beta_c (\text{Amenity}_i \times \text{Cycle}_{t,c}) +\varepsilon_{i,t}$, where $\alpha_t$ signifies time fixed effects and $\text{Cycle}_{t,c}$ is an indicator equal to 1 if period $t$ falls in cycle $c$ (and 0 otherwise). Housing market cycles span the 1970s, 1980s, 1990s, 2000--2007, 2008--2012, 2013--2019, and the 2020s. The far-right coefficients show the estimates for the full sample using the equation:  $\Delta \ln(P_{i,t}) = \alpha_t + \beta \text{Amenity}_i + \varepsilon_{i,t}$. Plotted coefficients are scaled by 100 and can be interpreted as annual percent changes in house prices per one standard deviation increase in Natural Amenity Demand. Confidence intervals are computed using $\pm 2$ robust standard errors clustered by CBSA.}}

\end{figure}

\clearpage



\clearpage
\input{lu_appendix.tex}

\end{appendix}

\end{document}
